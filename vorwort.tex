\chapter{Vorwort}

Der Inhalt dieses Skripts bildet die Grundlage der
Weiterbildungsveranstaltung der Deutschschweizerische
Mathematik-Kommission vom 8. September 2022, welche ich an der
Universität Basel gehalten habe.

Ich bin dankbar um die Meldung von Fehlern und Ungenaugigkeiten an
meine Email-Adresse auf der Titelseite.

Hier ist eine Liste von weiterführender Literatur
\begin{itemize}
\item Das Buch von Silverman und Tate~\cite{SilvermanTate}
  ist ein guter Einstieg in die Theorie von
  elliptischen Kurve.  Sie beweisen den Satz von Mordell (unten zitiert
  als Satz~\ref{satz:mordell}) in einem wichtigen Spezialfall.
\item Das Buch von Silverman~\cite{Silverman:AEC} ist weiter
  fortgeschritten. Etwas Vertrautheit mit  algebraischer Geometrie
  und  algebraischer Zahlentheorie wird vorausgesetzt. Es gibt eine
  Fortsetzung, ebenfalls von Silverman~\cite{Silverman:Adv}, welche
  Themen wie ``komplexe Multiplikation'' behandelt. 
\end{itemize}

\section{Notation}

Wir verwenden die folgenden Konventionen.

\begin{itemize}
\item Die Menge der natürlichen Zahlen ist $\IN = \{1,2,3,\ldots\}$.

\item Wir wählen eine Nullstelle von $X^2+1$ in $\IC$ und bezeichnen sie
  mit $\sqrt{-1}$.
  
\end{itemize}

%%% Local Variables:
%%% TeX-master: "main"
%%% End:
