\chapter{Kongruente Zahlen}

\begin{definition}
  Eine natürliche Zahl $n\in\IN$ heisst \emph{kongruente
    Zahl}\index{Kongruente Zahl}, falls
  $n$ die Fläche eines rechtwinkligen Dreiecks mit rationalen
  Seitenlängen ist. 
\end{definition}

\begin{bemerkung}
  Per Definition ist $n$ genau dann eine kongruente Zahl, wenn es positive
  $a,b,c\in\IQ$ gibt, so dass $a^2+b^2=c^2$ und $n = ab/2$. 
\end{bemerkung}

\begin{beispiele}\leavevmode
  \begin{enumerate}
  \item [(i)]
    Die Zahl $n=6$ ist kongruent, da es ein rechtwinkliges Dreieck mit Seitenlängen
    $a=3,b=4,c=5$  gibt und da $6=3\cdot 4/2$. 
  \item[(ii)]
    Für jede kongruente Zahl $n$ und jede rationale Zahl
    $\lambda\not=0$ ist $\lambda^2 n$ eine kongruente Zahl, sofern es
    eine ganze Zahl ist. Um das zu
    beweisen, dürfen wir $\lambda > 0$ annehmen. Nach Voraussetzung
    ist $n=ab/2$ mit $a^2+b^2=c^2$. Also ${a'}^2+{b'}^2={c'}^2$, wobei
    $a'=\lambda a, b'=\lambda b, c'=\lambda c$ wiederum rational.
    Die Fläche des entsprechenden rechtwinkligen Dreiecks ist
     $a'b'/2
    = \lambda^2 ab/2 = \lambda^2 n$. Diese Zahl ist kongruent, falls sie ganz ist.
  \item[(iii)]
    Aus (i) und (ii) folgt, dass es unendlich viele kongruente Zahlen
    gibt, denn $\{6\lambda^2 : \lambda\in\IN\}$ besteht aus kongruente
    Zahlen.
    Weiterhin ist jede kongruente Zahl ein rationales Vielfaches
    einer quadratfreien kongruenten Zahl.

  \item[(iv)] Es gilt $(3/2)^2 +(20/3)^2 = (41/6)^2$  und damit ist
    $n=5= (3/2)\cdot(20/3)/2$ kongruent.
    
  \item[(v)] Auch $n=7$ ist kongruent wegen $(35/12)^2 +
    (24/5)^2=(337/60)^2$.

  \end{enumerate}  
\end{beispiele}

Für jedes Tripel $(a,b,c)$ mit $a,b,c$ positive rationalen Zahlen mit
$a^2+b^2=c^2$ existieren $u>v>0$ rational mit $a =
u^2-v^2,b=2uv,c=u^2+v^2$. Hieraus können wir auf systematische Art
alle kongruente Zahlen produzieren. Es gilt folgt folgende Aussage.

\begin{lemma}
  Die Menge der kongruenten Zahlen ist
  $$
  \{ n \in \IN : \text{ es gibt $u>v>0$ in $\IQ$ mit $n=(u^2-v^2)uv$ }
  \}.
  $$
\end{lemma}


Ist $n=1$ eine kongruente Zahl? Die Antwort ist nein und dies wurde
von Pierre de Fermat bewiesen.

\begin{satz}[Fermat]
  \label{satz:fermat}
  Eins  ist keine kongruente Zahl.
\end{satz}
\begin{proof}
  TODO
\end{proof}

\begin{korollar}
  Die reelle Zahl $\sqrt{2}$ ist irrational. 
\end{korollar}
\begin{proof}
  Es gilt $a^2+b^2=2^2$ und $ab/2=1$ für $a=b=\sqrt 2$. Aber $1$ ist
  nicht eine kongruente Zahl wegen Fermats Satz. Damit kann $\sqrt 2$
  nicht rational sein.
\end{proof}

Eine klassische Fragestellung ist das folgende Problem.

\begin{problem}
  Gegeben sei eine natürliche Zahl $n$. Ist $n$ eine kongruente Zahl oder
  nicht? 
\end{problem}

Es ist heute kein Algorithmus bekannt, der entscheidet, ob eine
gegebene natürliche Zahl kongruent ist oder nicht. Daher kennen wir
keinen systematischen Zugang zu der Frage oben.

%In der Definition von kongruente Zahl kommen 

\begin{lemma} \label{lem:congruentelliptic}
  Sei $n\in\IN$. 
  \begin{enumerate}
  \item
  Seien $a,b,c\in \IQ$ mit $a^2+b^2=c^2,a\not=c$ und $n = ab/2$. Wir definieren
  \begin{equation*}
    x = \frac{nb}{c-a} \quad\text{und}\quad
    y = \frac{2n^2}{c-a}. 
  \end{equation*}
  Dann gilt $y^2 = x^3-n^2 x$ und $y\not=0$.
  \item Seien $x,y\in\IQ$ mit $y^2 = x^3-n^2x$. Falls $y\not=0$, dann
    ist $n$ eine kongruente Zahl.
  \end{enumerate}
\end{lemma}
\begin{proof}
  Teil (i) ist eine direkt Rechnung. Es gilt
  \begin{equation*}
    y^2 = \frac{4n^4}{(c-a)^2} = \frac{a^4b^4}{4(c-a)^2} 
  \end{equation*}
  und
  \begin{alignat*}1
    x^3 -n^2 x &= n^3 \frac{b^3}{(c-a)^3} - n^3 \frac{b}{c-a}
    = n^3 \frac{b}{c-a}\left(\frac{b^2}{(c-a)^2}-1\right)\\
    &= n^3 b\frac{b^2-(c-a)^2}{(c-a)^3}
    = \frac{a^3b^4}{8}\frac{b^2-(c-a)^2}{(c-a)^3}.
  \end{alignat*}
  Weiterhin gilt  $$
  y^2 - (x^3-n^2x) = \frac{a^3b^4}{8(c-a)^3} \left(
    2a(c-a) - b^2+(c-a)^2\right) =
  \frac{a^3b^4}{8(c-a)^3} (c^2-a^2-b^2)=0,
  $$
  was für (i)  zu zeigen.

  Für den Beweis von (ii) setzen wir
  \begin{equation*}
    a = \left|\frac{n^2-x^2}{y}\right|, \quad
    b = \left|\frac{2nx}{y}\right|, \quad\text{und}\quad
    c = \left|\frac{n^2+x^2}{y}\right|. 
  \end{equation*}
  Eine direkt Rechnung zeigt $a^2+b^2=c^2$. Weiterhin gilt
  \begin{equation*}
    \frac{ab}{2} = \frac{|(n^2-x^2)(2nx)|}{2y^2} = \frac{2n |n^2x -
      x^3|}{2y^2} = \frac{n |-y^2|}{y^2} = n.
  \end{equation*}

  Da $a,b,c$ nicht negative rationale Zahlen sind, reicht es zu
  zeigen, dass $abc\not=0$. Es gilt $c = (n^2+x^2)/|y| \ge n^2/|y|>0$.

  Es gilt $y^2 = (x^2-n^2)x\not=0$. Daraus folgt 
  $x^2-n^2\not=0$ und $x\not=0$. Also folgt $a\not=0$ und $b\not=0$. 

  Es folgt, dass $n$ eine kongruente Zahl ist. 
\end{proof}

Für jede natürliche Zahl $n\in\IN$ definiert
die Lösungsmenge der kubischen Gleichung
\begin{equation}
  \label{eq:ellipticcong}
  Y^2 = X^3-n^2X  
\end{equation}
eine Kurve in der reelle (oder komplexen) Ebene. Von besonderem
Interesse sind die \emph{rationalen Punkte} dieser Kurve, d.h. Punkte,
deren Koordinaten rational sind.

Die Punkte $(0,0),(\pm n,0)$ liegen augenscheinlich auf der Kurve für
jedes $n$. Gibt es mindestens ein weiterer rationaler Punkt, d.h. ein
Punkt dessen Ordinate nicht verschwindet, so ist $n$ eine kongruente
Zahl.
Weiterhin ist die Umkehrung auch wahr. 


Die Gleichung (\ref{eq:ellipticcong}) ist ein Spezialfall einer
Weierstrass-Gleichung, welche in grösserer Allgemeinen eine elliptische Kurve
definiert.

Den Satz von Fermat, Satz~\ref{satz:fermat}, lässt sich wie folgt
umformulieren.

\begin{theorem}[Fermat -- Version 2]
  \label{satz:fermat2}
  Die rationalen Punkten der Lösungsmenge von $Y^2 = X^3-X$ in der Ebene ist
  $$
  \{(0,0),(\pm 1,0)\}.
  $$
\end{theorem}

%%% Local Variables:
%%% TeX-master: "main"
%%% End:

