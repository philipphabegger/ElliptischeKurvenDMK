\chapter{Komplexe Multiplikation}

Eines der faszinierendsten Aspekte der Theorie elliptischer Kurven ist
die Möglichkeit komplexer Multiplikation. Diese soll nur an einigen
Beispielen veranschaulicht werden.

Wir beginnen mit einer Weierstrass-Gleichung $E : Y^2=X^3+aX+b$ mit
$a,b$ in einem beliebigen Körper $K$. Wie um (\ref{eq:amultP}) beschrieben,
können wir Punkt $P\in E(K)$ mit ganzen Zahlen multiplizieren, z.B.
$2P = P+P$ und $3P = P+P+P$ etc. Dies ist  in jeder Gruppe möglich.

\begin{beispiel}
  Wir betrachten die elliptische Kurve die durch  $E: Y^2 = X^3+X$
  definiert ist. Der Grundkörper $K$ sei nun $\IC$.
  Sei $(x,y)\in\IC^2$ eine Nullstelle von $Y^2 = X^3+X$. Dann gilt
  \begin{equation*}
    (\sqrt{-1} y)^2 = -y^2 = -(x^3+x) = (-x)^3+(-x). 
  \end{equation*}
  Also ist $(-x,\sqrt{-1}y)$ wieder eine Nullstelle der
  Weierstrass-Gleichung. 
  Für jeden Punkt $P\in E(\IC)$ definiert man
  \begin{equation*}
    I(P) = \left\{
      \begin{array}{ll}
        (-x, \sqrt{-1}y)         &: P = (x,y) \\
        \cO &: P = \cO.
      \end{array}\right.
  \end{equation*}
  Damit erhalten wir eine Selbstabbildung $I\colon E(\IC)\rightarrow
  E(\IC)$.

  Verkettet man $I$ mit sich selbst, so erhält man 
  \begin{equation*}
    I(I(P)) = (x,-y)
  \end{equation*}
  falls $P\not=\cO$. Also ist $I\circ I$ die Inversionsabbildung
  $-\colon E(\IC)\rightarrow E(\IC)$.

  In geeigneter Notation gilt $I^2 = I\circ I = -1$. 

  Man kann nun überprüfen, z.B. von Hand, dass $I$ mit dem
  Gruppengesetz kompatibel ist. D.h. es gilt
  \begin{equation*}
    I(P+Q) = I(P)+I(Q) \quad\text{für alle}\quad P,Q\in E(\IC). 
  \end{equation*}

  Schliesslich kann man weiter und für $\alpha,\beta\in\IZ$ eine
  weiter Selbstabbildung von $E(\IC)$ gemäss
  \begin{equation*}
    (\alpha + I \beta)(P) = (\alpha \cdot P) + (\beta \cdot I(P))
  \end{equation*}
  definieren. 

  Wie identifizieren die Menge $\{\alpha + I\beta :
  \alpha,\beta\in\IZ\}$ der eben definierten Selbstabbildung mit den
  Gaussschen Zahlen
  \begin{equation*}
    \IZ[\sqrt{-1}] = \{\alpha + \sqrt{-1}\beta : \alpha,\beta\in\IZ\}.
  \end{equation*}

  Wir erhalten eine Verknüpfung
  \begin{equation*}
    \IZ[\sqrt{-1}]\times E(\IC)\rightarrow E(\IC)
  \end{equation*}
  welche es uns erlaubt, Punkte in $E(\IC)$ mit Gausssche Zahlen zu
  multiplizieren. Für festes $\gamma\in\IZ[\sqrt{-1}]$ ist der
  Ausdruck $\gamma\cdot P$ ein komplizierter Bruch von Polynomen in
  den Koordinaten von $P$.

  Nicht jede elliptische Kurve erlaubt komplexe Multiplikation. In der
  Tat gibt es (in einem geeigneten Sinn) nur eine elliptische Kurve,
  die komplexe Multiplikation mit $\IZ[\sqrt{-1}]$ wie oben erlaubt.
\end{beispiel}

\begin{aufgabe}
  Überprüfen Sie mit der Definition, dass in der Notation des
  Beispiels oben $I(P+Q) = I(P)+I(Q)$ für alle $P,Q\in E(\IC)$ gilt.
\end{aufgabe}

Es gibt auch eine elliptische Kurven, die ``komplexe Multiplikation'' mit
dem Ring der Eisenstein Zahl
\begin{equation*}
  \left\{\alpha +\beta e^{2\pi\sqrt{-1}/6} : \alpha,\beta\in\IZ\right\}
\end{equation*}
hat. Im Allgemeinen taucht jede ``quadratische Ordnung'', darunter
z.B. auch
\begin{equation*}
  \left\{\alpha +\beta \frac{1+\sqrt{-163}}{2} : \alpha,\beta\in\IZ\right\}
\end{equation*}

Die Theorie der ``komplexen Multiplikation'' von elliptischen Kurven und deren
Verallgemeinerung begann im 19ten Jahrhundert und spielt auch noch
heute in der Grundlagenforschung eine zentrale Rolle~\cite{quanta:ao}.

Komplexe Multiplikation liefert auch eine überzeugende Erklärung,
wieso  
\begin{equation*}
  e^{\sqrt{163}\pi} = 262537412640768743,9999999999992500725971\ldots
\end{equation*}
fast die ganze Zahl  $640320^3 + 744$ ist.

%%% Local Variables:
%%% TeX-master: "main"
%%% End:
