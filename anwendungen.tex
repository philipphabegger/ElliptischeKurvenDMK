\chapter{Anwendungen}

In diesem Abschnitt untersuchen wir die folgenden
zwei Anwendungen der Theorie
elliptischer Kurven:

\begin{itemize}
\item Der Diffie-Hellman Schlüsselaustausch mit elliptischen Kurven
\item Lenstras (probabilistisches)
  Faktorisierungsverfahren für natürliche Zahlen
\end{itemize}

Für beide Anwendungen arbeiten wir nicht, wie in den ersten
Abschnitten, mit elliptischen Kurven über dem Körper $\IQ,\IR$ oder
$\IC$. Sondern wir werden mit elliptischen Kurven über einem endlichen
Körper oder gar einem endlichen Ring arbeiten. 

\section{Diffie-Hellman Schlüsselaustausch}

Der Diffie-Hellman Schlüsselaustausch liefert eine Lösung für das
folgende Problem.

Zwei Personen, hier $A$ und $B$ genannt, können nur über einem offenen
(und unsicheren!)
Kanal miteinander kommunizieren. Dies könnte z.B. eine
nicht-abhörsichere Telefonleitung sein oder die beiden kommunizieren
über Postkarten. Beide möchten sich auf ein gemeinsames Geheimnis
einigen. In den Anwendungen ist dieses gemeinsame Geheimnis z.B. der
Schlüssel für ein symmetrisches Verschlüsselungsverfahren wie 
AES. Sobald dieser gemeinsame Schlüssel beiden vorliegt, können A
und B  verschlüsselt miteinander kommunizieren. Beim
Bilden des gemeinsamen Schlüssels müssen A und B davon ausgehen, dass
Unbekannte zuhören. Es soll schlussendlich verhindert werden, dass
diese keinen Rückschlüsse auf das Geheimnis machen können.

Der Schlüsselaustausch funktioniert wie folgt.

\bigskip
\textbf{Schritt 1.} A und B legen a priori 
eine grosse Primzahl $p$ und damit einen endlichen Körper $\IF_p$ fest.
Sie einigen sich weiterhin auf eine elliptische Kurve
$$E: Y^2 = X^3+aX+b$$
wobei $a,b\in\IF_p$ und auf einen Punkt $P\in
E(\IF_p)\ssm\{\cO\}$. Die Sicherheit wird gewährleistet, wenn $p$
``gross'' ist und wenn die Ordnung von $P$ als Element der abelschen
Gruppe auch eine grosse  Primzahl.  
\footnote{Der einfachheitshalber arbeiten wir hier
  mit nicht mit langen Weierstrass-Gleichungen. In Anwendungen kommt
  z.B. die Primzahl 
 $p = 2^{255}-19$ und die elliptische Kurve
  $E: Y^2 = X^3+486662X^2+X$ mit langer Weierstrass-Gleichung vor. Der
rationale Punkt $P$ hat die Form $(9,*)$. Er erzeugt eine Untergruppe
von $E(\IF_p)$ dessen Ordnung die Primzahl $\#E(\IF_p)/8$ ist.}
Die gesamte Information $(p,E,P)$ ist an dieser Stelle rein
öffentlich. 
A und B müssen davon ausgehen, dass dritte Zugriff auf
diese Information haben.
Sie gilt sogar standardisierte Wahlen für  dieses Tripels in den Implementationen.

\bigskip
\textbf{Schritt 2.} In diesem Schritt wählt A per Zufall eine
natürliche Zahl $a$, die optimalerweise teilerfremd zur Ordnung von
$P$ ist. \emph{Die Zahl $a$ muss geheim bleiben}, nur A kennt sie. A berechnet nun den Punkt
$$
a\cdot P = \underbrace{P+P+P+\cdots + P}_{\text{$a$ mal}}.
$$

Diese Berechnung lässt sich wie folgt effizient gestalten.
Ist die Entwicklung von $a$ zur Basis $2$ durch $\sum_{i} a_i 2^i$ mit
$a_{i}\in \{0,1\}$ gegeben, so gilt
$$
a\cdot P =\sum_{i : a_i=1} 2^i \cdot P.$$
Also lässt sich $a \cdot P$ rein aus der  Addition $E(\IF_p)\times
E(\IF_p)\rightarrow E(\IF_p)$  und Iterationen der
Verdoppelungsabbildung  $2\colon E(\IF_p)\rightarrow E(\IF_p)$
bestimmen. 

\bigskip
\textbf{Schritt 3.} B macht das gleiche und wählt eine geheime natürliche Zahl
$b$. Daraufhin berechne $B$ den Punkt $b\cdot P \in E(\IF_p)$.

\bigskip
\textbf{Schritt 4.} Bis an hin wurde noch keine Information zwischen A
und B ausgetauscht (bis auf die Wahl von $(p,E,P)$ die öffentliche
Information ist.)
In Schritt 4 schickt A den Punkt $a\cdot P$ an B. Gleichzeitig schickt
B den Punkt
$b\cdot P$ an A. Dieser Informationsaustausch geschieht auf dem
öffentlichen und unsicheren Kanal. 


\bigskip
\textbf{Schritt 5.} Zu diesem Zeitpunkt besitzt A die Information $a$
und $b\cdot P$. Nun berechnet A den neuen Punkt
\begin{equation*}
  a\cdot (b\cdot P) \in E(\IF_p).
\end{equation*}
Auf der anderen Seite des Kanals berechnet B den Punkt
\begin{equation*}
  b\cdot (a\cdot P) \in E(\IF_p).
\end{equation*}

Nun sind wir am Ziel. Da $E(\IF_p)$ eine Gruppe ist,
gilt
$$
a\cdot (b\cdot P) = (ab)\cdot P = (ba) \cdot P = b\cdot (a\cdot P),$$
diese Gleichung wurde bereits in (\ref{eq:actionofZ}) angesprochen.

Das gemeinsame Geheimnis ist der Punkt $(ab)\cdot P$. Diese kann als
Grundlage für die Festlegung eines Schlüssel für ein symmetrisches
Verfahren genutzt werden.

\bigskip

Dieses Verfahren ist zur Zeit sicher, da es  keinen  effizienten Weg
gibt, den Wert $a$ (modulo $\mathrm{ord}(P)$)
aus $a\cdot P$ zu rekonstruieren. Diese Problem nennt sich
\emph{diskreter Logarithmus}.

In der Praxis ist $\mathrm{ord}(P)$
von der Grössenordnung $2^{256}$. Einfaches ``Absuchen'' von $a$ ist
nicht praktikabel.

Dennoch ist es nicht ausgeschlossen, dass es einen noch unbekannten
und effizienten Zugang zum diskreten Logarithmus gibt. Dabei bedeutet
``effizient'' ein Algorithmus der mit hoher Wahrscheinlichkeit $a$
produziert und zwar in $(\log p)^C$ Rechenschritt für eine Konstante
$C$.

Im Jahr 1994 hat Peter Shor~\cite{shor} einen ``Quantum-Algorithmus'' entwickelt,
welcher den Diffie-Hellman Schlüsselaustausch unsicher macht, sollte
ein hinreichend mächtiger Quantencomputer zur Verfügung stehen.
Kurzum, die zukünftige Bedeutung des Diffie-Hellman
Schlüsselaustausches ist offen.

\section{Lenstras Verfahren}

Gegeben sei eine natürliche zusammengesetzte Zahl $n\ge 2$.
Die Sicherheit des kryptographischen Verfahren RSA beruht auf der (angeblichen)
Schwierigkeit einen Primfaktor von $n$ zu finden.\footnote{Im
  konkreten Fall von RSA ist $n$ ein Produkt $pq$ aus zwei
  verschiedenen Primzahlen
  der Grössenordnung $2^{2048}$.}


Da $n$ zusammengesetzt ist, besitzt $n$ einen Primzahl $p$, so dass
$p\le \sqrt{n}$. Durch simples ausprobieren kann man mit ca. $\sqrt n$
ggT-Operationen $p$ finden. Aber für $n\cong 2^{2048}$ sind das ca.
$2^{1024}> 10^{300}$ Operationen. 

Lenstras Faktorisierungsverfahren ist ein probabilistischer
Algorithmus, um in Laufzeit ca. $e^{\sqrt{2\log n}\sqrt{\log\log n}}$
einen Primfaktor von $n$ zu finden. ``Probabilistisch'' bedeutet, dass
der Algorithmus mit ``grossen Wahrscheinlich'' zum Ziel führt. D.h. es
ist nicht mathematisch gesichert, dass der gesuchte Faktor gefunden
wird. Aber dennoch ist es aus sicherheits-theoretischen Überlegungen
wichtig, solche Algorithmen bei der Wahl eines kryptographischen
Systems zu berücksichtigen.

In der gleichen Arbeit~\cite{shor} hat Shor gezeigt, dass das
Faktorisierungsproblem für einen hinreichen starken Quantencomputer
in polynomialer Zeit (in $\log n$) bewältigbar ist.

Lentsras Faktorisierungsverfahren beruht auf der Theorie
elliptischer Kurven. Nun müssen wir jedoch einen Schritt weiter gehen
als bis an hin. Wir betrachten Weierstrass-Gleichungen mit
Koeffizienten in einem Ring, hier typischerweise $\IZ/n\IZ$. Da
$n$ in der Problemstellung gerade keine Primzahl ist, ist $\IZ/n\IZ$
kein Körper, es ist ein kommutativer Ring mit (nicht-trivialen)
Nullteiler. Wir werden nicht die Theorie elliptischer Kurven über
einem Ring entwickeln,\footnote{Solche Objekte heissen
  \emph{elliptische Schemata} oder \emph{abelsche Schemata}}
sondern  \textit{ad hoc} so rechnen, wie wir das in Abschnitt
\ref{kap:ek} über einem Körper beschrieben haben. Dabei sind einige
Vorkehrungen nötig, um Nullteiler zu berücksichtigen. Es sind aber
gerade die Nullteiler von $\IZ/n\IZ$, welche Rückschlüsse auf die
Primfaktoren von $n$ schliessen lassen.

Wir werden die folgende Notation verwenden. Gegeben sei $n\in\IN$. Eine ganze Zahl $a\in\IZ$
 repräsentiert eine Nebenklasse
$$
\overline a = a+n\IZ\in \IZ/n\IZ.
$$
Also verwenden wir $\overline a$ um die Nebenklasse zu beschreiben. 

Gegeben seien $a,b\in\IZ$ und die Gleichung $E : Y^2= X^3+\overline a
X + \overline b$ mit Koeffizienten in $\IZ/n\IZ$. Wir können die
Diskriminante
$$\Delta_E = \overline{-2^4(4a^3+27b^2)}\in\IZ/n\IZ$$ definieren. Aber die Bedingung $\Delta_E\not=0$
reicht in dieser Situation nicht aus, um $E$ als ``elliptische Kurve''
bezeichnen zu dürfen, den $\IZ/n\IZ$ besitzt Nullteiler. Wir brauchen
die stärkere Bedingung $\Delta_E \in (\IZ/n\IZ)^\times$, dabei
bezeichnet $(\IZ/n\IZ)^\times$ die Elemente in $\IZ/n\IZ$ die sich
multiplikativ invertieren lassen.\footnote{Für $n$ eine Primzahl ist $\IZ/n\IZ$
ein Körper und die Bedingung ist mit $\Delta_E\not=0$ gleichbedeutend.}
Es gilt
\begin{equation}
  \label{eq:ggTDeltaE}
  \Delta_E \in (\IZ/n\IZ)^\times\Longleftrightarrow
  \mathrm{ggT}(-2^4(4a^3+27b^2),n) = 1.
\end{equation}

In der Praxis lässt sich der ggT sehr effizient mittels Teilung
mit Rest berechnen. 
Diese Aussage ist für unser Anliegen interessant.  Sollte $\Delta_E$
\emph{keine} Einheit sein, dann haben die ``nicht reduzierte''
Diskriminante $-2^4(4a^3+27b^2)$ und $n$ einen gemeinsamen Teiler
$d>1$. Nun gibt es zwei Fälle:

\begin{itemize}
\item \textbf{Fall 1: Es gilt  $d=n$.}  In diesem Fall haben wir leider
  nichts gewonnen
\item \textbf{Fall 2: Es gilt  $d<n$.}
  Dann ist $d$ ein echten Teiler
  von $n$. Um ein Primfaktor von $n$ zu finden, reicht es, einen
  Primfaktor von $d$ zu finden. Da $d\le n/2$ hat sich das Problem durch
  diesen Schritt exponentiell vereinfacht. Ist $n=pq$ das Produkt verschiedener
  Primzahlen $p$ und $q$ (wie in RSA), so gilt sogar $d=p$ oder $d=q$.
\end{itemize}

Nun beschreiben wir eine vereinfachte Version von Lentras Verfahren.
Gegeben sie $n=pq\in\IN$
das Produkt zweier Primzahlen $p\not=q$. 

\bigskip
\textbf{Schritt 1}. Wir wählen zufällig $\overline a,\overline
x,\overline y\in\IZ/n\IZ$, repräsentiert durch $a,x,y\in \{0,\ldots,n-1\}$. 
Dann definieren wir
$$
b = y^2 - x^3 - ax\in \IZ.
$$
Das Paar $(\overline x,\overline y)$ ist eine Nullstelle von $Y^2 -
(X^3+\overline a X +\overline b)$.

\bigskip
\textbf{Schritt 2}. Wir berechnen $\Delta = -2^4(4a^3+27b^2)\in\IZ$.
Falls $\mathrm{ggT}(\Delta,n)>1$, d.h. $\overline\Delta$ ist
\emph{keine} Einheit in $\IZ/n\IZ$, so verwenden wir die Überlegung
direkt unterhalb von (\ref{eq:ggTDeltaE}).
Im (unwahrscheinlichen) Fall 1 kehren wir zurück zu Schritt 1 und wählen eine neue Kurve.
Alternativ kann man Fall 1 a priori ausschliessen, wenn man die
Repräsentanten $a,x,y\in \{0,\ldots,n\}$ klein genug in Funktion von
$n$ wählt, um $|\Delta|<n$ zu erzwingen.

Im Fall 2 muss $d=p$ oder $d=q$ gelten. Wir sind fertig, da ein
Primfaktor von $n$ berechnet wurde. Hier ist entscheidend, dass sich
der ggT schnell berechnen lässt.

Also können wir für das weitere Vorgehen annehmen, dass
$\Delta_E\in(\IZ/n\IZ)^\times$. Damit definiert
$E : Y^2 - (X^3+\overline a X + \overline b)$ eine elliptische Kurve
mit Koeffizienten in $\IZ/n\IZ$.

\bigskip
\textbf{Schritt 3}. Wir betrachten das Paar $(\overline x,\overline y)$
als Punkt $P$ in $E(\IZ/n\IZ)$. 
Die in Abschnitt~\ref{kap:ek}, Abschnitt \ref{sec:verknuepfung},
beschriebene Verknüpfung kann man
ad hoc auf die  
Punkte in  $E(\IZ/n\IZ)$ anzuwenden.
Unser Ziel ist es, ein Vielfaches $k \cdot P$ zu berechnen, dabei ist
$k\in\IN$ eine natürliche Zahl.

Um das effizient zu gestalten, entwickeln wir $k$, wie beim
Diffie-Hellman Schlüsselaustausch, zur Basis $2$. D.h. wir schreiben
$k=\sum_{i} k_i 2^i$ mit
$k_{i}\in \{0,1\}$. Wir möchten 
$$
\sum_{i : k_i=1} 2^i \cdot P
$$
bestimmen.

Insgesamt müssen wir auf $E(\IZ/n\IZ)$   zwei Punkte addieren können.
Da aber $\IZ/n\IZ$ kein Körper ist, darf nicht durch ein Element
ungleich Null dividiert werden. Es gibt  Nullteiler. 
Z.B. im Unterfall 1a wird bei der Konstruktion der Steigung $m$ durch
die Differenz $x_1-x_2$. Oder im Unterfall 2a, bei der Verdoppelung,
wird durch $2y$ geteilt.

Teilen ist jedoch nur erlaubt, wenn das entsprechende Element in
$(\IZ/n\IZ)^\times$ liegt. D.h. bei jeder Teilung müssen wir
überprüfen, dass der ggT eines Repräsentanten in $\IZ$, von $x_1-x_2$ oder $2y$,
zu $n$ gleich $1$ ist. Ist der ggT gleich $1$, so dürfen wir teilen
und die Berechnung durchführen.
Ist der ggT $d$ jedoch $>1$ so sind wir wieder in einer
Fallunterscheidungen wie unterhalb von (\ref{eq:ggTDeltaE}).

Sollte $d=n$ gelten haben wir ``Pech''. Wir müssen das Verfahren neu
bei Schritt 1 mit einer neuen Wahl von $\overline a,\overline
x,\overline y$ starten. Es scheint schwierig, diesen unglücklichen
Ausgang a priori auszuschliessen. Deshalb ist dieser Algorithmus
``probabilistisch''.

Gilt jedoch $1<d<n$, so muss (da $n=pq$) $d=p$ oder $d=q$. In diesem
Fall sind wir fertig, da ein Primfaktor gefunden wurde.

\bigskip

Die genaue Wahl von $k$ spielt für die Analyse des Algorithmus eine
wichtig Rolle auf die wir hier nicht eingehen werden. Es stellt sich
als vorteilhaft heraus, für $k$ ein Produkt von vielen ``kleinen''
Primzahlen zu wählen.

Eine einfache Implementation dieses Verfahrens ist als SageMath Skript
auf der GitHub repository in
\begin{center}
\texttt{sage/primfaktorisierung\_lenstra.ipynb}  
\end{center}
 zu finden. Hier wurde $k=10!$ festgelegt.





%%% local Variables:
%%% TeX-master: "main"
%%% End:

