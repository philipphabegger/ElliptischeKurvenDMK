\chapter{Anwendungen}

In diesem Abschnitt untersuchen wir die folgenden
zwei Anwendungen der Theorie
elliptischer Kurven:

\begin{itemize}
\item Der Diffie-Hellman Schlüsselaustausch mit elliptischen Kurven
\item Lenstras (statistisches)
  Faktorisierungsverfahrung für natürliche Zahlen
\end{itemize}

Für beide Anwendungen arbeiten wir nicht, wie in den ersten
Abschnitten, mit elliptischen Kurven über dem Körper $\IQ,\IR$ oder
$\IC$. Sondern wir werden mit elliptischen Kurven über einem endlichen
Körper oder gar einem endlichen Ring arbeiten. 

\section{Diffie-Hellman Schlüsselaustausch}

Der Diffie-Hellman Schlüsselaustausch liefert eine Lösung für das
folgende Problem.

Zwei Personen, hier $A$ und $B$ genannt, können nur über einem offenen
(und unsicheren!)
Kanal miteinander kommunizieren. Dies könnte z.B. eine
nicht-abhörsichere Telefonleitung sein oder die beiden kommunizieren
über Postkarten. Beide möchten sich auf ein gemeinsames Geheimnis
einigen. In den Anwendungen ist dieses gemeinsame Geheimnis z.B. der
Schlüssel für ein symmetrisches Verschlüsselungsverfahren wie 
AES. Sobald dieser gemeinsame Schlüssel beiden vorliegt, können A
und B also ungestört, d.h. verschlüssel miteinander kommunizieren. Bei
Bilden des gemeinsamen Schlüssels müssen A und B davon ausgehen, dass
Unbekannte zuhören. Es soll schlussendlich verhindert werden, dass
diese keinen Rückschlüsse auf das Geheimnis machen können.

Der Schlüsselaustausch funktioniert wie folgt.

\bigskip
\textbf{Schritt 1.} A und B legen a priori 
eine grosse Primzahl $p$ und damit einen endlichen Körper $\IF_p$ fest.
Sie einigen sich weiterhin auf eine elliptische Kurve
$$E: Y^2 = X^3+aX+b$$
wobei $a,b\in\IF_p$ und auf einen rationalen Punkt $P\in
E(\IF_p)\ssm\{\cO\}$. Die Sicherheit wird gewährleistet, wenn $p$
``gross'' ist und wenn die Ordnung of $P$ als Element der abelschen
Gruppe auch eine grosse  Primzahl.  
\footnote{Der Einfachheitshalber arbeiten wir hier
  mit nicht mit langen Weierstrass-Gleichungen. In Anwendungen kommt
  z.B. die Primzahl 
 $p = 2^{255-19}$ und die elliptische Kurve
  $E: Y^2 = X^3+486662X^2+X$ mit langer Weierstrass-Gleichung vor. Der
rationale Punkt $P$ hat die Form $(9,*)$. Er erzeugt eine Untergruppe
von $E(\IF_p)$ dessen Ordnung die Primzahl $\#E(\IF_p)/8$ ist.}
Die gesamte Information $(p,E,P)$ ist an dieser Stelle rein
öffentlich. A und B müssen davon ausgehen, dass dritte Zugriff auf
diese Information haben.

\bigskip
\textbf{Schritt 2.} In diesem Schritt wählt A per Zufall eine
natürliche Zahl $a$, die optimalerweise teilerfremd zur Ordnung von
$P$ ist. \emph{Die Zahl $a$ muss geheim bleiben}, nur A kennt sie. A berechnet nun den Punkt
$$
a\cdot P = \underbrace{P+P+P+\cdots + P}_{\text{$a$ mal}}.
$$

Diese Berechnung lässt sich wie folgt effizient gesalten.
Ist die Entwicklung von $a$ zur Basis $2$ durch $\sum_{i} a_i 2^i$ mit
$a_{i}\in \{0,1\}$, so gilt
$$
a\cdot P =\sum_{i : a_i=1} 2^i \cdot P.$$
Also lässt sich $a \cdot P$ rein aus der  Addition $E(\IF_p)\times
E(\IF_p)\rightarrow E(\IF_p)$  und Iterationen der
Verdoppelungsabbildung  $2\colon E(\IF_p)\rightarrow E(\IF_p)$
bestimmen. 

\bigskip
\textbf{Schritt 3.} B macht das gleiche und wählt eine geheime natürliche Zahl
$b$. Daraufhin berechne $B$ den Punkt $b\cdot P \in E(\IF_p)$.

\bigskip
\textbf{Schritt 4.} Bis anhin wurde noch keine Information zwischen A
und B ausgetausch (bis auf die Wahl von $(p,E,P)$.)
In Schritt 4 schickt A den Punkt $a\cdot P$ and B und B schickt
$b\cdot P$ an A. Der Informationsaustausch geschieht auf dem
öffentlichen und unsicheren Kanal. 


\bigskip
\textbf{Schritt 5.} Zu diesem Zeitpunkt besitzt A die Information $a$
und $b\cdot P$. Nun berechnet A den neuen Punkt
\begin{equation*}
  a\cdot (b\cdot P) \in E(\IF_p).
\end{equation*}
Auf der anderen Seite des Kanals berechnet B den Punkt
\begin{equation*}
  b\cdot (a\cdot P) \in E(\IF_p).
\end{equation*}

Nun sind wir am Ziel. Da $E(\IF_p)$ eine Gruppe ist,
gilt
$$
a\cdot (b\cdot P) = (ab)\cdot P = (ba) \cdot P = b\cdot (a\cdot P).$$

Das gemeinsame Geheimnis ist der Punkt $(ab)\cdot P$. Diese kann als
Grundlage für die Festlegung eines Schlüssel für ein symmetrisches
Verfahren genutzt werden.

Dieses Verfahren ist zur Zeit sicher, da es  keinen  effizienten Weg
gibt, den Wert $a$ (modulo $\mathrm{ord}(P)$)
aus $a\cdot P$ zu rekonstruieren. Diese Problem nennt sich
\emph{diskreter Logarithmus}.

In der Praxis ist $\mathrm{ord}(P)$
von der Grössenordnung $2^{256}$. Einfaches ``Absuchen'' von $a$ ist
nicht praktikabel.

Dennoch ist es nicht ausgeschlossen,
dass es einen noch unbekannten und
effizienten Zugang zum diskreten Logarithmus gibt. Dabei bedeutet
``effizient'' ein Algorithmus der mit hoher Wahrscheinlichkeit $a$
produziert und zwar in $(\log p)^C$ Rechenschritt für eine Konstante
$C$.

Im Jahr 1994 hat Peter Shor einen ``Quantum-Algorithmus'' entwickelt,
welcher den Diffie-Hellman Schlüsselaustausch unsicher macht, sobald
man einen hinreichend mächtigen Quantumcomputer bauen kann.
Kurzum, die zukunftige Bedeutung des Diffie-Hellman
Schlüsselaustausches ist offen.  


\section{Lenstras Verfahren}

%%% Local Variables:
%%% TeX-master: "main"
%%% End:

