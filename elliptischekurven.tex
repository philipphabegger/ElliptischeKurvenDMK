\chapter{Elliptische Kurven}

In diesem Abschnitt werden elliptische Kurven ad hoc eingeführt.
Teil des Datums einer elliptischen Kurve ist ein Grundkörper $K$. Dies
ist a priori ein beliebiger Körper. Aber für uns sind die wichtigsten
Wahlen von Grundkörper die rationalen Zahlen $\IQ$, die reellen Zahlen
$\IR$ und die komplexen Zahlen $\IC$. Für Anwendungen in der
Krytographie spielen  die endlichen Körper eine zentrale Rolle.


Es sei angemerkt, dass die Theorie elliptische Kurven im
arithmetischen Fall, d.h. für den  $K=\IQ$ und verwandte Körper, ihre
volle Tiefe entfaltet. Falls der Grundkörper, so wie $\IC$,
algebraisch abgeschlossen ist, verschwinden die arithmetischen
Aspekte.


\section{Definition der Elliptischen Kurve}

\begin{definition}
  Sei $K$ wie oben ein Körper.
  Eine \emph{Weierstrass-Gleichung}\index{Weierstrass-Gleichung} ist eine Gleichung vom Typ
  \begin{equation}
    \label{eq:weierstrass}
    E: Y^2 = X^3 + aX+b
  \end{equation}
  mit Unbekannten $X,Y$ und Koeffizienten $a,b\in K$, welche die
  Bedingung
  \begin{equation}
    \label{eq:disccondition}
    \Delta_E = -2^4(4a^3+27b^2)\not=0
  \end{equation}
  erfüllt. Die Grösse $\Delta_E$, ein Element aus $K$, heisst
  \emph{Diskriminante}\index{Diskriminante einer Weierstrass-Gleichung}
  der Weierstrass-Gleichung $E$. Zur Weierstrass-Gleichung $E$ gehört
  das Weierstrass-Polynom\footnote{Diese Beizeichnung ist nicht
    standard.} $Y^2- (X^3+aX+b)$.
\end{definition}

\begin{beispiele}\leavevmode
  \begin{enumerate}
  \item [(i)] In (\ref{eq:ellipticcong}) hatten wir die Gleichung $Y^2 = X^3-n^2X$
    für $n\in\IN$ betrachtet. Es handelt sich um eine
    Weierstrass-Gleichung $E$, für $K=\IQ$ (oder jeden Körper, der $\IQ$
    enthält) mit $\Delta_E = -2^4 3^3 n^4$.
  \item[(ii)] Die Gleichung
    \begin{equation*}
      Y^2 = X^3,
    \end{equation*}
    definiert keine Weierstass-Gleichung, da (\ref{eq:disccondition})
    nicht erfüllt ist.
  \end{enumerate}
\end{beispiele}

\begin{bemerkung}
  Sei $K$ ein Körper der Charakteristik $2$, z.B. $K=\IF_2$, und
  $a,b\in K$. Dann ist $Y^2 = X^3+aX+b$ keine Weierstrass-Gleichung,
  da (\ref{eq:disccondition}) nicht erfüllt ist. In $K$ gilt $2=0$.

  Das ist  unbefriedigend, da für Anwendung in der Kryptographe
  endliche Körper 
  der Charakteristik $2$ wichtig sind. Um Charakteristik $2$ (und auch
  $3$) abzudecken, muss man die Definition von Weierstrass-Gleichung
  verallgemeinern. Wir werden dies hier nicht tun, da es etwas
  technisch ist. Kurzum reicht es mit den sogenannten langen
  Weierstrass-Gleichungen vom Typ
  \begin{equation*}
    Y^2 + a_1 XY + a_3 Y = X^3+a_2 X^2 + a_4 X + a_6
  \end{equation*}
  mit entsprechender (aber kompliziertere) Diskriminanten-Bedingung
  \begin{alignat*}1
    &-a_6 a_1^6 + a_4 a_3 a_1^5 + ((-a_3^2 - 12 a_6) a_2 + a_4^2) a_1^4 +
    (8 a_4 a_3 a_2 + (a_3^3 + 36 a_6 a_3)) a_1^3\\
    &+ ((-8 a_3^2 - 48 a_6) a_2^2 +
    8 a_4^2 a_2 +
    (-30 a_4 a_3^2 + 72 a_6 a_4)) a_1^2 \\
    &+ (16 a_4 a_3 a_2^2 +
    (36 a_3^3 + 144 a_6 a_3) a_2 - 96 a_4^2 a_3) a_1 \\
    &+ ((-16 a_3^2 -
    64 a_6) a_2^3 + 16 a_4^2 a_2^2 + (72 a_4 a_3^2 + 288 a_6 a_4) a_2
    \\
    & +
    (-27 a_3^4 - 216 a_6 a_3^2 + (-64 a_4^3 - 432 a_6^2))) \not=0.
  \end{alignat*}
  zu arbeiten.

  In Charakteristik $\not=2$ und $\not=3$ kann man jede lange
  Weierstrass-Gleichung mittels quadratisch und kubischem Ergänzen
  durch eine affine lineare Transformation auf eine
  Weierstrass-Gleichung umformen. 
\end{bemerkung}

Nun wollen wir die Bedingung (\ref{eq:disccondition}) rechtfertigen.
Die partiell Ableitung eines Polynoms ist formal über jedem Körper
definiert, es ist kein Grenzwertbegriff notwendig.

\begin{lemma}
  \label{lem:smooth}
  Sei $F$ das Weierstrass-Polynom einer Weierstrass-Gleichung
  (\ref{eq:weierstrass}). Sei $(x,y)\in K^2$ mit $F(x,y)\not=0$. Dann
  gilt
  \begin{equation*}
    \frac{\partial F}{\partial X}(x,y)\not=0 \qoq
    \frac{\partial F}{\partial Y}(x,y)\not=0.
  \end{equation*}
\end{lemma}
\begin{proof}
  Es gilt $F = Y^2 - X^3-aX-b$ und $-2^4(4a^3+27b^2)\not=0$.
  Insbesondere gilt $2\not=0$ in $K$. Weiterhin
  \begin{equation*}
    \frac{\partial F}{\partial Y} = 2Y. 
  \end{equation*}

  Sicher ist diese Ableitung $\not=0$ an jedem Punkt $(x,y)$ mit
  $y\not=0$.
  Es reicht also zu zeigen, dass $    \frac{\partial F}{\partial
    X}(x,0)\not=0$, falls $F(x,0)=0$.

  Nun gilt
  \begin{equation*}
    \underbrace{(X^3+aX+b)}_{=F}(288aX-432b) +
    \underbrace{(-3X^2-a)}_{= \partial F/\partial X}(96aX^2 - 144bX + 64a^2)
    = -2^4(4a^3+27b^2) \not=0
  \end{equation*}
  nach Voraussetzung.
  Wir substituieren $X$ durch $x$ und finden 
  $\frac{\partial F}{\partial
    X}(x,0)\not=0$ da $F(x,0)$.  
\end{proof}

\begin{bemerkung}
  \label{bem:tangente}
  Im Fall $K=\IR$ können wir dieses letzte Lemma geometrisch wie folgt
  interpretieren.
  Sei $(x,y)\in\IR^2$ eine Nullstelle von $F$, dem Weierstrass-Polynom
  einer Weierstrass-Gleichung. Wir setzen
  $$\alpha = -\frac{\partial F}{\partial Y}(x,y) \quq
  \beta = \frac{\partial F}{\partial X}(x,y).$$
  Dann ist $(\alpha,\beta)$ nicht der Nullvektor.

  Aus dem Satz von der impliziten Funktion folgt, wir die
  Nullstellenmenge von $F$ in $\IR^2$ in der Nähe von $(x,y)$ durch
  den Graph einer glatten Funktion (nach einem möglichen Koordinatentausch)
  ausdrücken können. 

  Weiterhin ist die Menge
  \begin{equation*}
    T = \{(x+\alpha t,y+\beta t) : t\in\IR\}
  \end{equation*}
  eine Gerade durch $(x,y)$. Wir definieren $f(t) =
  F(x+t\alpha,y+t\beta)$ für alle $t\in\IR$. 
  Dann gilt 
  \begin{equation*}
    \frac{d\gamma}{dt}(t)
    = \underbrace{F(x,y)}_{=0} + \left(\underbrace{\alpha \frac{\partial F}{\partial X}(x,y)
    +\beta \frac{\partial F}{\partial Y}(x,y)}_{=0}\right)t + \text{(Terme der
      Ordnung $\ge 2$ in $t$)}.
  \end{equation*}
  
  Also hat $t\mapsto F(x+t\alpha,y+t\beta)$ eine mehrfache Nullstelle
  bei $t=0$. Dies bedeutet, dass $T$ die Tangent an der
  Nullstellenmenge von $F$ ist.

  Zusammengefasst: die Gerade durch $(x,y)$ mit Richtungsvektor
  \begin{equation*}
    \left(-\frac{\partial F}{\partial Y}(x,y),\frac{\partial F}{\partial X}(x,y)\right)
  \end{equation*}
  liegt tangential an der Nullstellenmenge von $F$.
\end{bemerkung}

Die Schlussfolgerung von Lemma~\ref{lem:smooth} besagt, dass die
Nullstellenmenge einer Weierstrass-Gleichung an jedem Punkt einer
Bedingung genügt, welche sich zumindest im reellen Fall
als ``Glattheitsbedigung'' verstehen lässt.

\begin{definition}
  \label{def:tangentensteigung}
  Sei $F$ das Weierstrass-Polynom einer Weierstrass-Gleichung. Falls
  $(x,y)\in K$ und $y\not=0$, dann heisst
  \begin{equation*}
    \frac{\frac{\partial F}{\partial X}}{-\frac{\partial F}{\partial
        Y}}(x,y) = \frac{3x^2+a}{2y}
  \end{equation*}
  \emph{Tangentensteigung} an $(x,y)$. 
\end{definition}

Sei $m$ die Tangentensteigung an $(x,y)$.
So wie in Beispiel \ref{bem:tangente} zeigt man, dass
$F(x+t,y+mt)$ bei $t=0$ eine Nullstelle der Ordnung $\ge 2$ hat. 

\begin{beispiel}
  Die Lösungsmenge von $Y^2=X^3$ hat bei $(0,0)$ eine ``Spitze''.  Die
  Nullstellenmenge ist an diesem Punkt nicht glatt, vgl.
  Abbildung~\ref{fig:cusp} links.

  \begin{figure}
    \centering    
    \caption{Lösungsmenge von $Y^2=X^3$ (links) und $Y^2 = X^3-X$ (rechts)}
    \label{fig:cusp}
    \includegraphics[width=0.3\textwidth]{./plots/cusp.png}
    \includegraphics[width=0.3\textwidth]{./plots/smooth_n_equal_1.png}
  \end{figure}

  In der gleichen Abbildung rechts ist eine glatte Kurve zu sehen.
\end{beispiel}

\section{Das Gruppengesetz}

In diesem Abschnitt werden wir zeigen, wie man auf der Lösungsmenge
einer Weierstrass-Gleichung zusammen mit einem zusätzlichen Punkt,
eine Gruppenstruktur definieren kann. Die Konstruktion kann man rein
geometrisch veraschaulichen. Wie oben bezeichnet $K$ ein Körper in dem
sich alles abspielt (sofern nicht anders vermerkt). 

\begin{definition}
  Sei $Y^2 = X^3+aX+b$ eine Weierstrass-Gleichung $E$.
  Wir definieren 
  \begin{equation*}
    E(K) = \{(x,y) \in K^2 : y^2 = x^3 + ax+b\} \cup \{\cO\}.
  \end{equation*}
  wobei $\cO$ zunächst ein weiteres Element ist, welches nicht in
  $K^2$ liegt. Man nennt $E(K)$ die \emph{Menge der $K$-rationalen Punkten
  von $E$}. 
\end{definition}

\begin{beispiel}
  Für die Weierstrass-Gleichung $E$ gegeben durch $Y^2 = X^3-X$ und
  für $K=\IQ$ besagt der 
  Satz von Fermat, Satz~\ref{satz:fermat2}, dass
  $$ E(\IQ) = \{(0,0), (\pm 1,0), \cO\} .$$
  aus vier Elementen besteht. 
\end{beispiel}

Wir werden zeigen, dass wir $E(K)$ mit der Struktur einer abelschen
Gruppe verstehen können.
Dazu müssen wir eine Verknüpfung $+\colon E(K)\times
E(K)\rightarrow E(K)$, eine Inversionsabbildung
$-\colon E(K)\rightarrow E(K)$, sowie ein neutrales Element in $E(K)$
produzieren.



\subsection{Die Verknüpfung}
\label{sec:verknuepfung}

Sei $E$ eine Weierstrass-Gleichung gegeben durch $Y^2 = X^3+aX+b$ mit
$a,b\in K$. 
In einem ersten Schritt werden wir eine Verknüpfung
\begin{equation*}
  + \colon E(K)\times E(K) \rightarrow E(K)
\end{equation*}
definieren. Es ist üblich, die Verknüpfung mit ``$+$'' zu bezeichnen.
Später werden wir feststellen, dass die so konstruierte Gruppe eine
abelsche Gruppe ist. 

Seien $P$ und $Q$ Elemente von $E(K)$. Es folgt eine Aufspaltung in
verschiendene Fälle. Wir werden bereits in der Konstruktion
überprüfen, dass die Verküpfung $+$ die Bedingung
$$
P+Q=Q+P\quad\text{für alle}\quad P,Q\in E(K) 
$$
erfüllt. Hieraus werden wir folgern, dass die Gruppe $E(K)$ abelsch
ist.  

\medskip
\textbf{Fall 1: Die Menge $\{P,Q,\cO\}$ hat drei Elemente.} In diesem
Fall gilt $P = (x_1,y_1)$ und $Q = (x_2,y_2)$. Weiterhin haben wir
$P\not=Q$.
Daher gibt es genau eine Gerade $G$ durch $P$ und $Q$.

Wir unterscheiden zwei Unterfälle. 

\textbf{Unterfall 1a. Es gilt $x_1\not=x_2$.} In diesem Fall hat die
Gerade $G$ (endliche) Steigung $m = \frac{y_2-y_1}{x_2-x_1}\in K$. In
anderen Worten, sie hat die Gesalt
\begin{equation*}
  G = \{(x,mx+q) : x\in K \}
\end{equation*}
mit $q=y_1-m x_1$. Vgl. Abbildung \ref{fig:unterfall1a} darin ist $G$
gestrichelt.

\begin{figure}
  \centering    
  \caption{$E: Y^2 = X^3-5^2 X,P = (-4,6),Q = (\frac{1681}{144},-\frac{62279}{1728})$}
  \label{fig:unterfall1a}
  \includegraphics[width=0.3\textwidth]{./plots/unterfall1a.png}
\end{figure}
 
Daher sind $P$ und $Q$ Schnittpunkte der Gerade $G$ mit der Menge
$E(K)\ssm \{\cO\}$. Mit Vielfachheiten gezählt hat die Gerade $G$
jedoch drei Schnittpunkte mit der Lösungsmenge von $Y^2=X^3+aX+b$. Wir
können dies wie folgt direkt überprüfen. Dazu defineren wir
das Polynom
\begin{equation*}
  A = (mX+q)^2 - (X^3+aX+b) =-X^3+m^2X +\text{(Terme von Grad $\le 1$ in $X$)}
  \in K[X]. 
\end{equation*}

Das Polynom $A$ hat Grad $3$  und wir kennen bereits zwei verschiedene
Nullstellen: $x_1,x_2\in K$. Daher lässt sich $A$ durch $X-x_1$ und
$X-x_2$ teilen. Es gilt also $A = -(X-x_1)(X-x_2)(X-x_3)$, dabei muss
$x_3$ als wiederum ein Element in $K$ sein, denn es gilt $x_1+x_2+x_3
= m^2$, also
\begin{equation*}
  x_3 = \left(\frac{y_2-y_1}{x_2-x_1}\right)^2-x_2-x_1.
\end{equation*}
Nach Konstruktion ist das Paar $(x_3,y_3)$ mit  $y_3=mx_3+q$ eine
Nullstelle von $Y^2-(X^3+aX+b)$. Weiterhin ist auch $(x_3,-y_3)$ eine
Nullstelle.

In Unterfall 1a definieren wir
\begin{equation*}
  P+Q = (x_3,-y_3) \in E(K)\ssm \{\cO\}.
\end{equation*}

Sind wir in  Unterfall 1a, so ist auch das Paar $(Q,P)$ in Unterfall
1a. Es gilt $P+Q=Q+P$, da die Gerade und sowohl $(m,q)$ unabhängig von
der Reihenfolge von $P,Q$ ist. 

\textbf{Unterfall 1b. Es gilt $x_1=x_2$.} Nun liegt die Gerade $G$
senkrecht zur Abszisse, vgl. Abbildung~\ref{fig:unterfall1b}. Es gilt
\begin{equation*}
  y_1^2 = x_1^3+a x_1 +b = x_2^3+a x_2 +b = y_2^2
\end{equation*}
und daher $y_1=-y_2$ da $P\not=Q$. Nun stehen wir vor einem
Dilemma, die Gerade $G$ hat keine weiteren Schnittpunkte mit $E(K)$ in
der Ebene $K^2$. Jetzt kommt uns der Punkt $\cO$ zur Hilfe.

In Unterfall 1b definieren wir
\begin{equation*}
  P+Q = \cO \in E(K).
\end{equation*}

Vertauscht man $P,Q$ so bleiben wir in Unterfall 1b und es gilt
trivialerweise $P+Q=Q+P$.  

\begin{figure}
  \centering    
  \caption{$E: Y^2 = X^3-5^2 X,P = (-4,6),Q = (-4,-6)$}
  \label{fig:unterfall1b}
  \includegraphics[width=0.3\textwidth]{./plots/unterfall1b.png}
\end{figure}

\medskip
\textbf{Fall 2: Die Menge $\{P,Q,\cO\}$ hat zwei Elemente.} Auch hier
gibt es mehrere Unterfälle.

\textbf{Unterfall 2a: Es gilt $P=Q\not=\cO$ und $P=(x,y)$ mit $y\not=0$.}
%Wir schreiben $P=Q=(x,y)$ mit $x,y\in K$.

Die Gerade, welche in Fall 1 konstruiert wurde, ist nun nicht
eindeutig bestimmt.

Etwas Intuition schafft die folgende Überlegung. Wir
versetzen uns in die reelle Welt und ersetzen den Punkte $Q$ durch
eine Folge von Punkten $(Q_n)_{n\in\IN}$ aus $E(\IR)\ssm\{Q,\cO\}$, dessen
Koordinaten für $n\rightarrow\infty$ gegen $Q=P$ konvergieren.
Die Gerade $G_n$ durch $P$ und $Q_n$ ist wohldefiniert und die Summe
$P+Q_n$ lässt sich mit der Vorschrift aus Fall 1 bestimmen.
Anschaulich nähert sich die Gerade $G_n$ der Tangente and
$E(\IR)\ssm\{\cO\}$ durch den Punkte $P$. Vgl.
Abbildung~\ref{fig:unterfall2a}. 

\begin{figure}
  \centering    
  \caption{$E: Y^2 = X^3-5^2 X,P = Q=(-4,6)$}
  \label{fig:unterfall2a}
  \includegraphics[width=0.3\textwidth]{./plots/unterfall2a.png}
\end{figure}

Für einen allgemeinen Körper können wir zwar nicht mit solchen
Grenzbegriffen argumentieren, aber wir haben einen Ersatz für die
Tangente in Bemerkung~\ref{bem:tangente} und Definition
\ref{def:tangentensteigung}
gefunden.


Für $y\not=0$ dürfen wir  Definition \ref{def:tangentensteigung}
anwenden. Das weitere Vorgehen ist vergleichbar mit Unterfall 1a.
Wir setzen
$$m =  \frac{3x^2+a}{2y}\in K\quq q = y-mx\in K.$$
Das Polynom
$$
A = (mX+a)^2 - (X^3+aX+b) \in K[X]
$$
hat nur eine Nullstelle der Ordnung $\ge 2$ in $x$. Dies lässt sich
rein formal mit Hilfe der Definition von $m$ überprüfen.
Also gilt $A = -(X-x)^2(X-x')$ für ein $x'\in K$. Dabei gilt
\begin{equation*}
  x' = m^2 - 2x = \left(\frac{3x^2+a}{2y}\right)^2 - 2x. 
\end{equation*}

Der Punkt $(x',y')$ mit $y'=mx'+q$ liegt in $E(K)$. Wie in Unterfall
1a liegt $(x',-y')$ auch in $E(K)$. 
In Unterfall 2a
definieren wir
\begin{equation*}
  P+Q = P+P = \left(x',-y'\right).
\end{equation*}

Trivialierweise gilt $P+Q=Q+P$ in diesem Unterfall.

\textbf{Unterfall 2b: Es gilt $P=Q\not=\cO$ und $P=(x,0)$.}
Im Fall $y=0$ ist die Tangente durch $P$ and $E(K)$ parallel zur
Ordinate. Wir 
definieren
\begin{equation*}
  P+Q = P+P = (x,0)+(x,0) = \cO. 
\end{equation*}
Trivialierweise gilt $P+Q=Q+P$ in diesem Unterfall.

\textbf{Unterfall 2c: Es gilt $P=\cO\not=Q$.}
Wir definieren 
\begin{equation*}
  P+Q = \cO + Q = Q. 
\end{equation*}


\textbf{Unterfall 2d: Es gilt $P\not=\cO=Q$.}
Dieser Fall ist analog zu Unterfall 2b, wir definieren hier
\begin{equation*}
  P+Q = P+\cO = P. 
\end{equation*}
Vergleicht mah Unterfälle 2b und 2c, so sehen wir $P+Q=Q+P$, falls
$P=\cO$ oder $Q=\cO$. 


\medskip
\textbf{Fall 3: Die Menge $\{P,Q,\cO\}$ hat ein Elemente.} Es gilt
$P=Q=\cO$ und wir definieren
\begin{equation*}
  P+Q = \cO+\cO = \cO. 
\end{equation*}
Trivialerweise gilt $P+Q=Q+P$. 

\subsection{Die Inversionsabbildung}

Sei $P\in E(K)$. Hier gibt es nur zwei Fälle.

{\textbf Fall 1. Es gilt $P\not=\cO$.} In diesem Fall gilt $P=(x,y)$.
Wegen $y^2 = x^3+ax+b$ ist auch $(x,-y)$ eine Lösung der Weierstrass-Gleichung.
Wir  definieren
$$
-P = (x,-y)\in E(K)\ssm\{\c0\}.
$$

{\textbf Fall 2. Es gilt $P=\cO$.} 
Wir  definieren
$$
-P = \cO \in E(K).
$$

\subsection{Das neutral Element}
Es sollte nun nicht überraschen, dass wir $\cO$ als das neutrale
Element designieren.

\section{Überprüfung der Gruppenaxiome}

\begin{satz}
  Sei $E$ eine Weierstrass-Gleichung mit Koeffizienten in einen Körper
  $K$. Sei $\cdot$ die Verknüpfung aus
  Abschnitt~\ref{sec:verknuepfung}.
  Dann ist $(E(K),+,\cO)$ eine abelsche Gruppe. 
\end{satz}

Die Abbildung $+\colon E(K)\times E(K) \rightarrow E(K)$ ist
wohldefiniert. Weiterhin überprüfen wir mit der Hilfe von den  Unterfällen
1b, 2b und Fall 3 der Konstruktion, dass
$$  (-P)+P = \cO$$
für alle $P\in E(K)$.

Weiterhin ist $\cO + P = P $ für alle $P\in \cO$, dies folgt aus den
Unterfällen 2c, 2d und Fall 3 in der Konstruktion. 

Schliesslich muss noch die Assoziativität der Verknüpfung gezeigt
werden. Dies läuft auf die Gleichung
$$
P + (Q + R) = (P+Q)+R$$
für alle $P,Q,R\in E(K)$ hinaus.

Das ist ein nicht-trivialer Schritt den wir hier nicht beweisen
werden. Es gibt mehrere Ansätz die Assoziativität zu zeigen.
Naheliegend ist es, die Gleichheit mit der Definition direkt zu
überprüfen. Das ist im Prinzip möglich. Dazu müssen jedoch die vier
Verknüpfungen $Q+R, P+(Q+R), P+Q$ und $(P+Q)+R$ gebildet werden.
Pro Verknüpfung gibt es $7$ Fälle zu unterscheiden. D.h. ingesamt gibt
es $7^4= 2041$ Fälle. Obwohl einige Fälle trivialerweise stimmen, ist
ein systematisches Arbeiten mit viel Aufwand verbunden.


Es gibt einen weiteren Zugang zur Assoziativität über die sogenannte
 \emph{Picard-Gruppe}, einem  Objekt der algebraischen Geometrie
welches man $E$ zuordnen kann. Die Picard-Gruppe ist aus theoretischen
Überlegungen \textit{a priori} eine abelsche Gruppe. Die Idee ist nun, eine
bijektive Abbildung zwischen $E(K)$ und der Picard-Gruppe zu
konstruieren, welche die oben dargestellte Verknüpfung mit dem
Gruppengesetzt der Picard-Gruppe in Verbindung setzt.




Die Verknüpfung $E(K)\times
E(K)\rightarrow E(K)$ sowie die Inversion $E(K)\rightarrow E(K)$
werden mit der Hilfe von 
Quotienten von Polynomen mit Koeffizienten in $K$ beschrieben.


%%% Local Variables:
%%% TeX-master: "main"
%%% End:

